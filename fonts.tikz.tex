\begin{center}
\begin{tikzpicture}
    \begin{scope}[x={(current page.south east)},y={(current page.north west)}]
        \if\helplines1
        	\path[draw, help lines,xstep=.1,ystep=.1] (0,0) grid (\N, \N);
        \else
            \path[help lines,xstep=.1,ystep=.1] (0,0) grid (\N, \N);
        \fi
        \node[text width=0.4\pagewidth, align=justify, anchor= west](titleen) at (0.075, 0.5){\english{\normalsize The English text is set in \emph{IM Fell English}, a type cut by Christoffel van Dijck and Robert Granjon. In 1672, John Fell, Bishop of Oxford, bought them for the printer in Oxford.
		Nowadays the are not only the oldest surviving punches in England, but are still being used in new books.
         
        \doindent The digital incarnation was created by Igino Marini between 2000 and 2006.}};
		
		
		\node[text width=0.401\pagewidth, align=justify, anchor= west] (titlees) at (0.525, 0.5) {\spanish{\normalsize \oldstylenums El texto en español está tipografiado en una fuente grabada por  Jerónimo Antonio Gil en el siglo XVIII. Fue seleccionada para la edición de \emph{El Quijote} publicada por la Real Academia en \oldstylenums{1780}. 
		
		\doindent La versión digital, llamada \emph{Ibarra Real} fue resucitada por la Calcografía Nacional y José María Ribagorda en el año \oldstylenums{2005}.}};
    \end{scope}

\end{tikzpicture}
\end{center}