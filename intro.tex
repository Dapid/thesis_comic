\begin{center}
\begin{tikzpicture}
    \begin{scope}[x={(current page.south east)},y={(current page.north west)}]
        \if\helplines1
        	\draw[help lines,xstep=.1,ystep=.1] (0,0) grid (\N, \N);
        \else
            \path[help lines,xstep=.1,ystep=.1] (0,0) grid (\N, \N);
        \fi
        \node[text width=0.4\pagewidth, align=center, anchor=north west](titleen) at (0.075, 0.7){\english{\Huge  Introduction}};
		\node[text width=0.4\pagewidth, align=justify, below=1cm of titleen] {\english{This is an attempt to explain my thesis in a way that is accessible for everyone.
		
		\doindent The comic, just as the thesis, is structured in the following way:
		
		\begin{itemize}
		\item First, some background. I will explain what proteins are, why we care, and the problems I worked on.
		\item Then, the papers I published during my PhD. Brief summary of the content, and what I think is the most important message from each.
		\end{itemize}}};
		
		
		\node[text width=0.4\pagewidth, align=center, anchor=north west] (titlees) at (0.525, 0.72) {\spanish{\Huge Introdución}};
		\node[text width=0.4\pagewidth, align=justify, below =1cm of titlees] {\spanish{Este comic es un intento de explicar mi tesis de una forma accessible a todos.
		
		\doindent Al igual que la tesis que acompaña, el comic está dividido en dos partes:
		
		\begin{itemize}
		\item Primero, algo información básica. Explicaré qué son las proteínas y por qué nos importan, además de presentar los problemas en los que trabajé.
		\item Después, los artículos publicados. Un breve resumen de cada uno, explicando lo que, en mi opinión, es el mensaje más importante de cada uno.
		\end{itemize}}};
    \end{scope}

\end{tikzpicture}
\end{center}